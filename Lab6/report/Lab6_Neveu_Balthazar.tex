\documentclass[a4paper]{article}
\addtolength{\hoffset}{-2.25cm}
\addtolength{\textwidth}{4.5cm}
\addtolength{\voffset}{-3.25cm}
\addtolength{\textheight}{5cm}
\setlength{\parskip}{0pt}
\setlength{\parindent}{0in}

\usepackage[square,sort,comma,numbers]{natbib}
\usepackage{blindtext} % Package to generate dummy text
\usepackage{charter} % Use the Charter font
\usepackage[utf8]{inputenc} % Use UTF-8 encoding
\usepackage{microtype} % Slightly tweak font spacing for aesthetics
\usepackage{amsthm, amsmath, amssymb} % Mathematical typesetting
\usepackage{float} % Improved interface for floating objects
\usepackage{hyperref} % For hyperlinks in the PDF
\usepackage{graphicx, multicol} % Enhanced support for graphics
\usepackage{xcolor} % Driver-independent color extensions
\usepackage{pseudocode} % Environment for specifying algorithms in a natural way
\usepackage[mmddyy]{datetime} % Uses YEAR-MONTH-DAY format for dates

\usepackage{fancyhdr} % Headers and footers
\pagestyle{fancy} % All pages have headers and footers
\fancyhead{}\renewcommand{\headrulewidth}{0pt} % Blank out the default header
\fancyfoot[L]{} % Custom footer text
\fancyfoot[C]{} % Custom footer text
\fancyfoot[R]{\thepage} % Custom footer text
\newcommand{\note}[1]{\marginpar{\scriptsize \textcolor{red}{#1}}} % Enables comments in red on margin

\DeclareMathOperator*{\argmin}{arg\,min}

%----------------------------------------------------------------------------------------

\newcommand{\yourname}{Balthazar Neveu}
\newcommand{\youremail}{balthazarneveu@gmail.com}
\newcommand{\assignmentnumber}{6}

\begin{document}

\fancyhead[C]{}
\hrule \medskip
\begin{minipage}{0.295\textwidth} 
\raggedright
\footnotesize
\yourname \hfill\\
\youremail
\end{minipage}
\begin{minipage}{0.4\textwidth} 
\centering 
\large 
Lab session \# \assignmentnumber\\ 
\normalsize 
ALTEGRAD 2023\\ 
\end{minipage}
\begin{minipage}{0.295\textwidth} 
\raggedleft
\today\hfill\\
\end{minipage}
\medskip\hrule 
\bigskip


\section*{Code}

More info:
\href{https://github.com/balthazarneveu/MVA23_ALTEGRAD/#readme}{MVA ALTEGRAD Balthazar Neveu on Github}

\section{Section 1}
\subsection*{Question 1: $\mathbb{E}(d)^{p=0.2} = 4.8$ and $\mathbb{E}(d)^{p=0.4} = 9.6$}
We denote a graph $G = (V, E)$ where $v_i \in V$ denote the vertices or nodes and $E$ are the edges.
\subsubsection*{Preliminary remark : Overall statistics of edges}

The total number of undirected edges when the graph is dense is ${n\choose 2} =\frac{n(n-1)}{2}$
When $n=25$, the densest graph will have 300 edges (when $p=1$).
The Erdős–Rényi has in average $p * {n=25 \choose 2}$ edges.

\subsubsection*{Distribution of the degree $d$}
For a given node $v{i}$, the probability to be connected to $v_j$ where $i \neq j$ is a
Bernoulli distribution.
$$\mathbb{P}[(v_{i},v_{j}) \in E] \sim \mathbb{B}(p)$$ 
Degree of edge $v_{i}$ is the number of all connected nodes (number of edges).
It therefore follows a Binomial distribution. 
The maximum number of edges is $n-1$ ($d \leq n-1$)
The expectation of the degree $d$ is:
$$\mathbb{E}(d) = (n-1).p$$

\textit{Note: Computing the degree is equivalent to counting the number of flipped coins with a tail (1 toss of a coin with probability $p$ of having a tail is equivalent 
to get a connected 1 edge)}

\textit{Note: Obviously, degenerate case when $p=1$, the degree reaches the maximum as we'd expect.}

\subsubsection*{Numerical application}
$$\mathbb{E}(d)^{p=0.2} = 24*0.2 = 4.8$$
$$\mathbb{E}(d)^{p=0.4} = 24*0.4 = 9.6$$

\subsection*{Task 3}
\begin{verbatim}
Epoch: 0191 loss_train: 0.3165 acc_train: 85.39% time: 0.0434s
Optimization finished!
loss_test: 0.2615 acc_test: 88.89% time: 0.0466s
\end{verbatim}

% \bibliographystyle{plain}
% \bibliography{references}

\end{document}