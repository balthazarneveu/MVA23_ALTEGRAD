\documentclass[a4paper]{article} 
\input{style/head.tex}

%-------------------------------
%	TITLE VARIABLES (identify your work!)
%-------------------------------

\newcommand{\yourname}{Balthazar Neveu}
\newcommand{\youremail}{balthazarneveu@gmail.com}
\newcommand{\assignmentnumber}{2}

\begin{document}

%-------------------------------
%	TITLE SECTION (do not modify unless you really need to)
%-------------------------------
\input{style/header.tex}

%-------------------------------
%	ASSIGNMENT CONTENT (add your responses)
%-------------------------------
\section{Question 1:}
Assume $G = (V, E)$ is an undirected graph of n nodes without self-loops. $|V|=n$.
\subsection*{Number of edges}
The maximum number of edges is the cardinal of the set of possible combinations of 2 nodes chosen from n nodes.
which is equal to the binomial coefficient $\binom{n}{2} = \frac{n!}{2! (n-2)!}$
\begin{equation}\label{eq 1.1}
|E| \leq \frac{n*(n-1)}{2}
\end{equation}
This can also be viewed when writing the adjacency matrix of a complete graph.
\begin{itemize}
    \item  A matrix full of 1 has $n^2$ elements
    \item set the diagonal to 0 to remove the self loops. $n*(n-1)$ elements
    \item Divide by two since we consider an undirected graph.
\end{itemize}


In the code the property is verified through an assert.

\subsection*{Number of triangles}
The maximum number of triangles is $\binom{n}{3} = \frac{n!}{3! * (n-3)!} = \frac{n*(n-1)(n-2)}{6}$ 
when choosing the combination of 3 nodes from n nodes.


\section{Question 2:}
%------------------------------------------------

\bibliographystyle{plain}
\bibliography{references} % citation records are in the references.bib document

\end{document}