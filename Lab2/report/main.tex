\documentclass[a4paper]{article} 
\input{style/head.tex}
\newcommand{\yourname}{Balthazar Neveu}
\newcommand{\youremail}{balthazarneveu@gmail.com}
\newcommand{\assignmentnumber}{2}

\begin{document}

\input{style/header.tex}

\part[short]{Analyzing a Real-World Graph}

\section*{Coding results}

\subsection*{Task 1 : Basic graph statistics extraction}
\begin{verbatim}
{'total_edges': 25998,
 'total_nodes': 9877}
max number of edges 48772626
\end{verbatim}


\subsection*{Task 2 : Extract the connected components of the graph}
\begin{verbatim}
Graph has 429 connected components
Largest connected component:
{'total_edges': 24827,
 'total_nodes': 8638}
8638 total_nodes represent 87.46% of the graph
24827 total_edges represent 95.50% of the graph
\end{verbatim}


\subsection*{Task 3 : Statistics on the degrees of the nodes of the graph}
\begin{verbatim}
{'degree_of_nodes': {'max': 65,
                     'mean': 5.264351523742027,
                     'median': 3.0,
                     'min': 1}}
\end{verbatim}


\subsection*{Task 4 : Degree histogram plot}
\begin{verbatim}
\end{verbatim}\begin{figure}[ht]
        \centering
        \includegraphics[width=.6\textwidth]{figures/histogram_degree_of_nodes.png}
        \caption{Histogram of degrees of the nodes}
\end{figure}
\begin{verbatim}
\end{verbatim}


\subsection*{Task 5 : Global clustering coefficient}
\begin{verbatim}
Graph clustering coefficient: 0.284
\end{verbatim}


\section{Question 1:}
Assume $G = (V, E)$ is an undirected graph of n nodes without self-loops. $|V|=n$.
\subsection*{Number of edges}
The maximum number of edges is the cardinal of the set of possible combinations of 2 nodes chosen from n nodes.
which is equal to the binomial coefficient $\binom{n}{2} = \frac{n!}{2! (n-2)!}$
\begin{equation}\label{eq 1.1}
|E| \leq \frac{n*(n-1)}{2}
\end{equation}
This can also be viewed when writing the adjacency matrix of a complete graph.
\begin{itemize}
    \item  A matrix full of 1 has $n^2$ elements
    \item set the diagonal to 0 to remove the self loops. $n*(n-1)$ elements
    \item Divide by two since we consider an undirected graph.
\end{itemize}


In the code the property is verified through an assert.

\subsection*{Number of triangles}
The maximum number of triangles is $\binom{n}{3} = \frac{n!}{3! * (n-3)!} = \frac{n*(n-1)(n-2)}{6}$ 
when choosing the combination of 3 nodes from n nodes.


\section{Question 2 : 2 graphs having the same degree distribution $\not \implies$ isomorphic}

If two graphs have the same degree distribution, it does not imply that they are isomorphic to each other.
The simplest counter-example with degrees of nodes of 2 is presented 
in \ref{fig:graph_triangle_vs_three_single_loops},
it's a bit of a degenerate case as we're using the fact that a loop adds two to the degree in case of undirected graphs.
Below are four cases illustrating the counter example.
\ref{fig:graph_compare_triangle_hexagon} shows an hexagon versus 2 triangles 
which is the most pleasant example to visualize.



\begin{figure}[ht]
    \centering
    \includegraphics[width=.6\textwidth]{figures/graph_triangle_vs_three_single_loops.png}
    \caption{Graphs G1 is a triangle and G2 is made of 3 isolated nodes with a self loop.They have the same degree distribution (every node has a degree of 2)but are not isomorphic to each other.}
    \label{fig:graph_triangle_vs_three_single_loops}
\end{figure}



\begin{figure}[ht]
    \centering
    \includegraphics[width=.6\textwidth]{figures/graph_rect_vs_triangle_plus_single_loop.png}
    \caption{G1 is a rectangle, G2 is made of a triangle and a single node with a self loop.}
    \label{fig:graph_rect_vs_triangle_plus_single_loop}
\end{figure}



\begin{figure}[ht]
    \centering
    \includegraphics[width=.6\textwidth]{figures/graph_compare_triangle_hexagon.png}
    \caption{G1 is an hexagon, it has 6 edges. G2 has 2 separate triangles.All nodes have a degree of 2, G1 and G2 have the same degree histograms.But they are not isomorphic to each other}
    \label{fig:graph_compare_triangle_hexagon}
\end{figure}


\begin{figure}[ht]
    \centering
    \includegraphics[width=.6\textwidth]{figures/graph_compare_quad.png}
    \caption{Counter example where all nodes have a degree of 3.G1=(a-b, b-c, c-a, b-d, d-c, d-a) is a rectangle with its diagonalsG2=(u-v, u-u, v-v, w-x, w-w, x-x) are 2 segment where the end nodes have self loops}
    \label{fig:graph_compare_quad}
\end{figure}

\pagebreak
\section{Question 3 : n-cycle graphs}
\begin{itemize}
    \item Clustering coefficient of the triangle $C_3$ is $1=\frac{1}{1+0}$
    \item Clustering coefficient of cycle graphs $C_n$ where $n \geq 4$ is $0=\frac{0}{0+n}$
    as there are no closed triplets but only ($n$) open triplets. 
\end{itemize}
\begin{figure}[ht]
        \centering
        \includegraphics[width=.6\textwidth]{figures/cycle_graphs.png}
        \caption{Transitivity of $C_n$ cycle graphs becomes 0 if $n \geq 4$}
        \label{fig:cycle_graphs}
\end{figure}




\pagebreak

\part[short]{Community detection}

\subsection*{Task 6 : Toy example for spectral graph clustering}

\begin{figure}[ht]
    \centering
    \includegraphics[width=.6\textwidth]{figures/spectral_clustering_toy_example.png}
    \caption{G =union of 3 disjoint subgraphs (complete, cycle, star). Spectral graph clustering, k=3 clusters found correctly}
    \label{fig:spectral_clustering_toy_example}
\end{figure}

\begin{verbatim}
    ADJACENCY MATRIX:
    [[0 1 1 0 0 0 0 0 0 0 0 0 0]
     [1 0 1 0 0 0 0 0 0 0 0 0 0]
     [1 1 0 0 0 0 0 0 0 0 0 0 0]
     [0 0 0 0 1 0 1 0 0 0 0 0 0]
     [0 0 0 1 0 1 0 0 0 0 0 0 0]
     [0 0 0 0 1 0 1 0 0 0 0 0 0]
     [0 0 0 1 0 1 0 0 0 0 0 0 0]
     [0 0 0 0 0 0 0 0 1 1 1 1 1]
     [0 0 0 0 0 0 0 1 0 0 0 0 0]
     [0 0 0 0 0 0 0 1 0 0 0 0 0]
     [0 0 0 0 0 0 0 1 0 0 0 0 0]
     [0 0 0 0 0 0 0 1 0 0 0 0 0]
     [0 0 0 0 0 0 0 1 0 0 0 0 0]]
\end{verbatim}


\pagebreak
\begin{verbatim}
    LAPLACIAN:
    [[ 1.  -0.5 -0.5  0.   0.   0.   0.   0.   0.   0.   0.   0.   0. ]
     [-0.5  1.  -0.5  0.   0.   0.   0.   0.   0.   0.   0.   0.   0. ]
     [-0.5 -0.5  1.   0.   0.   0.   0.   0.   0.   0.   0.   0.   0. ]
     [ 0.   0.   0.   1.  -0.5  0.  -0.5  0.   0.   0.   0.   0.   0. ]
     [ 0.   0.   0.  -0.5  1.  -0.5  0.   0.   0.   0.   0.   0.   0. ]
     [ 0.   0.   0.   0.  -0.5  1.  -0.5  0.   0.   0.   0.   0.   0. ]
     [ 0.   0.   0.  -0.5  0.  -0.5  1.   0.   0.   0.   0.   0.   0. ]
     [ 0.   0.   0.   0.   0.   0.   0.   1.  -0.2 -0.2 -0.2 -0.2 -0.2]
     [ 0.   0.   0.   0.   0.   0.   0.  -1.   1.   0.   0.   0.   0. ]
     [ 0.   0.   0.   0.   0.   0.   0.  -1.   0.   1.   0.   0.   0. ]
     [ 0.   0.   0.   0.   0.   0.   0.  -1.   0.   0.   1.   0.   0. ]
     [ 0.   0.   0.   0.   0.   0.   0.  -1.   0.   0.   0.   1.   0. ]
     [ 0.   0.   0.   0.   0.   0.   0.  -1.   0.   0.   0.   0.   1. ]]
\end{verbatim}
We clearly observe the 3 blocks on the diagonal. When we perform the eigen decomposition
and keep the $d=3$ eigen vectors associated with the 3 lowest eigen values (0, 0, 0). 

$eigen values = [0, 0, 0, 1 , 1, 1, 1, 1, 1, 1.5, 1.5, 2, 2]$. 
The first 3 zeros correspond to the 3 disjoint subgraphs.


\begin{verbatim}
    MATRIX OF EIGEN VECTORS ASSOCIATED WITH THE THREE LOWEST EIGEN VALUES
    [[ 0.          0.         -0.57735027]
     [ 0.          0.         -0.57735027]
     [ 0.          0.         -0.57735027]
     [ 0.5         0.          0.        ]
     [ 0.5         0.          0.        ]
     [ 0.5         0.          0.        ]
     [ 0.5         0.          0.        ]
     [ 0.         -0.40824829  0.        ]
     [ 0.         -0.40824829  0.        ]
     [ 0.         -0.40824829  0.        ]
     [ 0.         -0.40824829  0.        ]
     [ 0.         -0.40824829  0.        ]
     [ 0.         -0.40824829  0.        ]]     
\end{verbatim}

\section{Question 4 : Spectral clustering}
TODO 
\section{Question 5 : Modularity}
TODO
\pagebreak
\part[short]{Graph classification}

\subsection*{Task 10 : Cycle and graph dataset creation}

\begin{figure}[ht]
        \centering
        \includegraphics[width=1.\textwidth]{figures/cycle_and_paths_dataset.png}
        \caption{Dataset is made of cycles $C_n$ and paths $P_n$}
        \label{fig:cycle_and_paths_dataset}
\end{figure}


\section{Question 6 : Shortest path kernel}
In \ref{fig:P4_shortest_paths_computation} and \ref{fig:S4_shortest_paths_computation}, we show 
how we progressively build the shortest paths graph for the path $P_4$ and the start $S_4$.
Then we simply compute the distribution of the edges weights (=length) to get the so called "feature map" $\phi$.
We get:
\begin{itemize}
    \item $\phi(P_4)=[3, 2, 1, 0 ...]^T$
    \item $\phi(S_4)=[3, 3, 0, 0 ...]^T$
    \item $k(P_4, P_4) = 3^2 + 2^2 + 1^2 = 14$
    \item $k(P_4, S_4) = K(S_4, P_4) = 3*3+2*3+1*0 = 15$
    \item $k(S_4, S_4) = 3^2 + 3^2 = 18$
\end{itemize}

\begin{figure}[ht]
        \centering
        \includegraphics[width=1.\textwidth]{figures/P4_shortest_paths_computation.png}
        \caption{Path graph $P_4$, $\phi(P_4)=[3, 2, 1, 0 ...]^T$}
        \label{fig:P4_shortest_paths_computation}
\end{figure}

\begin{figure}[ht]
    \centering
    \includegraphics[width=.6\textwidth]{figures/S4_shortest_paths_computation.png}
    \caption{Start graph $S_4$ , $\phi(S_4)=[3, 3, 0, 0 ...]^T$}
    \label{fig:S4_shortest_paths_computation}
\end{figure}

\pagebreak
\section*{Task 11: Finding isomorphic graphlets in subgraphs}
We show in \ref{fig:finding_isomorphic_graphlets_subgraphs} and \ref{fig:finding_isomorphic_graphlets_subgraphs_triangle}
how to compute the feature maps for each sampled triplet subgraph.
We repeat this sampling process $N=200$ times for each graph.
\begin{figure}[ht]
    \centering
    \includegraphics[width=.6\textwidth]{figures/finding_isomorphic_graphlets_subgraphs.png}
    \caption{Finding graphlet "$G_1$" (notation of the code) which is isomorphic with a randomly sampled subgraph made of 3 nodes.
    Feature vector is $[0, 1, 0, 0]^T$ for this subgraph}
    \label{fig:finding_isomorphic_graphlets_subgraphs}
\end{figure}

\begin{figure}[ht]
    \centering
    \includegraphics[width=.6\textwidth]{figures/finding_isomorphic_graphlets_subgraphs_triangle.png}
    \caption{Finding triangle graphlet "$G_3$" ( notation of the code) which is isomorphic with a randomly sampled subgraph made of 3 nodes?
    Feature vector is $[0, 0, 0, 1]^T$ for this subgraph}
    \label{fig:finding_isomorphic_graphlets_subgraphs_triangle}
\end{figure}
\pagebreak

\subsection*{Task 13 : SVM classifier comparison of kernels (graphlet vs shortest path)}
\begin{verbatim}
Graphlet based kernel, SVM classifier (trained with accuracy=50.56%) Test accuracy: 45.00%
Shortest path based kernel, SVM classifier (trained with accuracy=100.00%) Test accuracy: 85.00%
\end{verbatim}

Shortest path kernel based classifier is perfectly accurate on this dataset. Graphlet accuracy is much lower.
This is explained by the fact that when we use Graphlet feature, the discriminative triplets choice
involve choosing the endpoints of the Path $P_n$ which makes a difference with cycle graphs.


\section{Question 7 : Shortest path kernel}
TODO
%------------------------------------------------

\bibliographystyle{plain}
\bibliography{references}
\end{document}